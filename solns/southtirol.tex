\documentclass[10pt]{amsart}
%prepared in AMSLaTeX, under LaTeX2e
\addtolength{\oddsidemargin}{-.7in}
\addtolength{\evensidemargin}{-.7in}
\addtolength{\topmargin}{-.4in}
\addtolength{\textwidth}{1.3in}
\addtolength{\textheight}{1.0in}

\newcommand{\normalspacing}{\renewcommand{\baselinestretch}{1.05}
        \tiny\normalsize}

\usepackage{amssymb,fancyvrb,alltt,xspace}

\VerbatimFootnotes

\usepackage{fancyvrb}
\newcommand{\mfile}[1]{
\begin{quote}
\bigskip
%\VerbatimInput[frame=single]{#1}
\VerbatimInput[frame=single,label=\fbox{\normalsize \textsl{\,#1\,}},fontfamily=courier,fontsize=\small]{#1}
\end{quote}
}


% macros
\newcommand{\CC}{\mathbb{C}}
\newcommand{\Div}{\nabla\cdot}
\newcommand{\eps}{\epsilon}
\newcommand{\grad}{\nabla}
\newcommand{\ZZ}{\mathbb{Z}}
\newcommand{\ip}[2]{\ensuremath{\left<#1,#2\right>}}
\newcommand{\lam}{\lambda}
\newcommand{\lap}{\triangle}
\newcommand{\RR}{\mathbb{R}}

\newcommand{\prob}[1]{\bigskip\noindent\large\textbf{#1.}\normalsize }
\newcommand{\apart}[1]{\quad \textbf{(#1)} \quad }
\newcommand{\ppart}[1]{\medskip\noindent\textbf{(#1)} \quad }

\newcommand{\Matlab}{\textsc{Matlab}\xspace}
\newcommand{\Octave}{\textsc{Octave}\xspace}


\begin{document}
\Large\centerline{\textbf{Solutions}}

\medskip
\small
\centerline{\emph{These are brief solutions to some notes available at a small town in the Schnalstal valley.}}
\normalsize

\bigskip\bigskip

\thispagestyle{empty}
\normalspacing

\prob{1}  Start with two Taylor expansions around $x$:
    $$f(x\pm \Delta) = f(x) \pm f'(x) \Delta + \frac{1}{2} f''(x) \Delta^2 \pm \frac{1}{6} f'''(x) \Delta^3 + \dots$$
Subtracting the two:
    $$f(x+\Delta) - f(x-\Delta) = 2 f'(x) \Delta + \frac{1}{3} f'''(x) \Delta^3 + \dots$$
Solve for $f'(x)$:
    $$f'(x) = \frac{f(x+\Delta) - f(x-\Delta)}{2\Delta} - \frac{1}{6} f'''(x) \Delta^2 + \dots$$
This shows the first result.  For the second we expand to one more term and do the same basic steps, starting with adding the two expansions and then solving for $f''(x)$:
\begin{gather*}
    f(x\pm \Delta) = f(x) \pm f'(x) \Delta + \frac{1}{2} f''(x) \Delta^2 \pm \frac{1}{6} f'''(x) \Delta^3 + \frac{1}{24} f^{(4)}(x) \Delta^4 + \dots \\
    f(x + \Delta) + f(x - \Delta) = 2 f(x) + f''(x) \Delta^2 + \frac{1}{12} f^{(4)}(x) \Delta^4 + \dots \\
    f''(x) = \frac{f(x + \Delta) - 2 f(x) + f(x - \Delta)}{\Delta^2} - \frac{1}{12} f^{(4)}(x) \Delta^2 + \dots
\end{gather*}
This shows the second result.

\prob{2}  In problem \textbf{1} we had $x$ as base-point and we moved $\Delta$ to left and right.  Now we use $x+\Delta/2$ as base-point and move $\Delta/2$ to left and right:
\begin{gather*}
f(x) = f(x+\Delta/2) - \frac{1}{2} f'(x+\Delta/2) \Delta + \frac{1}{8} f''(x+\Delta/2) \Delta^2 - \frac{1}{48} f'''(x+\Delta/2) \Delta^3 + \dots \\
f(x+\Delta) = f(x+\Delta/2) + \frac{1}{2} f'(x+\Delta/2) \Delta + \frac{1}{8} f''(x+\Delta/2) \Delta^2 + \frac{1}{48} f'''(x+\Delta/2) \Delta^3 + \dots
\end{gather*}
Subtracting these and solving for $f'(x+\Delta x/2)$ give, in turn,
\begin{gather*}
f(x+\Delta) - f(x) = f'(x+\Delta/2) \Delta + \frac{1}{24} f'''(x+\Delta/2) \Delta^3 + \dots \\
f'(x+\Delta/2) = \frac{f(x+\Delta) - f(x)}{\Delta} - \frac{1}{24} f'''(x+\Delta/2) \Delta^2 + \dots
\end{gather*}
as desired.

\prob{3}

\mfile{heatwithloops.m}

\prob{4}  If $\Delta t < 0$ then $\mu = D \Delta t / \Delta x^2 < 0$ also.  Then $T_j^{n+1} = \mu T_{j+1}^n + (1 - 2 \mu) T_j^n + \mu T_{j-1}^n$ writes $T_j^{n+1}$ as a linear combination of the three ``old'' values, but with the coefficients of $T_{j+1}^n$ and $T_{j-1}^n$ both negative.  Though the sum of coefficients is still one (i.e.~$\mu + (1-2\mu) + \mu = 1$), this is not an average for any $\Delta t < 0$.

On the other hand, running
\begin{verbatim}
  >> heat(1.0,30,30,-0.001,20);
\end{verbatim}
which goes backward in time from $t=0$ to $t=-0.020$, produces an obviously nonsensical distribution of heat (not shown).

\prob{5}  Rewriting (35) for $T_j^{n+1}$ gives
    $$T_j^{n+1} \stackrel{\ast}{=} T_j^n + \mu T_{j+1}^{n+1} - 2 \mu T_j^{n+1} + \mu T_{j-1}^{n+1}$$
or
    $$(1 - 2\mu) T_j^{n+1} = T_j^n + \mu T_{j+1}^{n+1} + \mu T_{j-1}^{n+1}$$
or
    $$T_j^{n+1} = \frac{1}{1 - 2\mu} T_j^n + \frac{\mu}{1 - 2\mu} T_{j+1}^{n+1} + \frac{\mu}{1 - 2\mu} T_{j-1}^{n+1}.$$

In the last form we have written $T_j^{n+1}$ as an average of neighboring points.  That is, the coefficients are positive and add to one.  This allows a ``maximum principle'' proof of unconditional convergence, and thus unconditional stability; see reference [37] (= Morton \& Mayers).

An alternative proof of unconditional stability is via a von Neumann-Fourier argument, also in [37].  Specifically, we look at the evolution of waves on the space-time grid:
    $$T_j^n \stackrel{\dagger}{=} \lambda^n e^{i\omega (j\Delta x)}.$$
We are interested in how the growth/decay rate $\lambda$ depends on the spatial frequency $\omega$.  Numerical stability is the statement ``all waves (all $\omega$) decay ($|\lambda|<1$)''.  Unconditional numerical stability is when this occurs regardless of the values of $\Delta t$ and $\Delta x$.

In this case, if we start from the first form of the scheme $\ast$ above, and substitute \emph{ansatz} $\dagger$, and cancel as much as possible, we get
    $$\lambda = 1 + \mu \lambda e^{i\omega\Delta x} - 2 \mu \lambda + \mu \lambda e^{-i\omega\Delta x}$$
or (recall $\cos z = (e^{iz} + e^{-iz}) / 2$)
    $$\lambda = 1 - 2 \mu \lambda + 2 \mu \lambda \cos(\omega\Delta x)$$
or
    $$\lambda (1 + 2\mu (1 - \cos(\omega\Delta x)) = 1$$
or (recall $2 \sin^2 z = \cos(2z)$)
    $$\lambda = \frac{1}{1 + 4 \mu \sin^(\omega\Delta x)}.$$
The point about this last form is that $\lambda > 0$ is obvious \emph{and} $\lambda < 1$ is obvious.  In particular, the scheme makes \emph{all spatial waves (modes) decay}, regardless of the value of $\mu$, and thus regardless of the grid spacings $\Delta t$ and $\Delta x$.  This is unconditional stability.


\prob{6}  

\prob{7}  

\prob{8}  

\prob{9}  

\prob{10}  

\prob{11}  The time-dependent margin radius of the Halfar solution is $R(t) = R_0 (t/t_0)^{1/18}$, an expression which comes from setting $H(t,r)=0$ and solving.  The volume at time $t$ is found by integrating out to this radius.  Using polar coordinates and successive substitutions $s = (t/t_0)^{1/18}$ and then $u=r/(R_0 s)$ we have
\begin{align*}
V(t) &= \int_0^{2\pi} \int_0^{R(t)} H(t,r) \,r\,dr\,d\theta \\
     &= 2\pi \int_0^{R_0 (t/t_0)^{1/18}} H_0 \left(\frac{t_0}{t}\right)^{1/9} \left[1 - \left(\left(\frac{t_0}{t}\right)^{1/18} \frac{r}{R_0}\right)^{4/3}\right]^{3/7} \,r\,dr \\
     &= 2\pi H_0 R_0^2 \int_0^1 \left(1 - u^{4/3}\right)^{3/7} \,u\,du.
\end{align*}
At this point we are done!  The last expression is not known yet,\footnote{The integral can be easily approximated numerically: $\int_0^1 (1 - u^{4/3})^{3/7} \,u\,du \approx 0.31422$.  Thus $V(t) = C H_0 R_0^2$ for a value $C\approx 2$ which can be computed accurately to many digits, if desired.} but it is independent of $t$.
%>> quadl(@(u) u.*(1 - u.^(4/3)).^(3/7),0,1)
%ans =  0.31422
%>> quad(@(u) u.*(1 - u.^(4/3)).^(3/7),0,1)
%ans =  0.31422

\prob{12}  

\prob{13}  

\prob{14}  

\prob{15}  

\prob{16}  

\prob{17}  

\prob{18}  

\end{document}

