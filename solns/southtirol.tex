\documentclass[10pt]{amsart}
%prepared in AMSLaTeX, under LaTeX2e
\addtolength{\oddsidemargin}{-.7in}
\addtolength{\evensidemargin}{-.7in}
\addtolength{\topmargin}{-.4in}
\addtolength{\textwidth}{1.3in}
\addtolength{\textheight}{1.0in}

\newcommand{\normalspacing}{\renewcommand{\baselinestretch}{1.05}
        \tiny\normalsize}

\usepackage{amssymb,fancyvrb,alltt,xspace}

\VerbatimFootnotes

\usepackage{fancyvrb}
\newcommand{\mfile}[1]{
\begin{quote}
\bigskip
%\VerbatimInput[frame=single]{#1}
\VerbatimInput[frame=single,label=\fbox{\normalsize \textsl{\,#1\,}},fontfamily=courier,fontsize=\footnotesize]{#1}
\end{quote}
}


% macros
\newcommand{\CC}{\mathbb{C}}
\newcommand{\Div}{\nabla\cdot}
\newcommand{\eps}{\epsilon}
\newcommand{\grad}{\nabla}
\newcommand{\ZZ}{\mathbb{Z}}
\newcommand{\ip}[2]{\ensuremath{\left<#1,#2\right>}}
\newcommand{\lam}{\lambda}
\newcommand{\lap}{\triangle}
\newcommand{\RR}{\mathbb{R}}

\newcommand{\prob}[1]{\bigskip\noindent\large\textbf{#1.}\normalsize }
\newcommand{\apart}[1]{\quad \textbf{(#1)} \quad }
\newcommand{\ppart}[1]{\medskip\noindent\textbf{(#1)} \quad }

\newcommand{\Matlab}{\textsc{Matlab}\xspace}
\newcommand{\Octave}{\textsc{Octave}\xspace}


\begin{document}
\scriptsize
\noindent September 2014 \hfill Ed Bueler

\vspace{10mm}

\Large\centerline{\textbf{Solutions}}

\medskip
\small
\centerline{\emph{These are reasonably complete solutions to some notes available at a small town in the Schnalstal valley.}}
\normalsize

\bigskip\bigskip

\thispagestyle{empty}
\normalspacing

\prob{1}  Start with two Taylor expansions around $x$:
    $$f(x\pm \Delta) = f(x) \pm f'(x) \Delta + \frac{1}{2} f''(x) \Delta^2 \pm \frac{1}{6} f'''(x) \Delta^3 + \dots$$
Subtracting the two:
    $$f(x+\Delta) - f(x-\Delta) = 2 f'(x) \Delta + \frac{1}{3} f'''(x) \Delta^3 + \dots$$
Solve for $f'(x)$:
    $$f'(x) = \frac{f(x+\Delta) - f(x-\Delta)}{2\Delta} - \frac{1}{6} f'''(x) \Delta^2 + \dots$$
This shows the first result.  For the second we expand to one more term and do the same basic steps, starting with adding the two expansions and then solving for $f''(x)$:
\begin{gather*}
    f(x\pm \Delta) = f(x) \pm f'(x) \Delta + \frac{1}{2} f''(x) \Delta^2 \pm \frac{1}{6} f'''(x) \Delta^3 + \frac{1}{24} f^{(4)}(x) \Delta^4 + \dots \\
    f(x + \Delta) + f(x - \Delta) = 2 f(x) + f''(x) \Delta^2 + \frac{1}{12} f^{(4)}(x) \Delta^4 + \dots \\
    f''(x) = \frac{f(x + \Delta) - 2 f(x) + f(x - \Delta)}{\Delta^2} - \frac{1}{12} f^{(4)}(x) \Delta^2 + \dots
\end{gather*}
This shows the second result.

\prob{2}  In problem \textbf{1} we had $x$ as base-point and we moved $\Delta$ to left and right.  Now we use $x+\Delta/2$ as base-point and move $\Delta/2$ to left and right:
\begin{gather*}
f(x) = f(x+\Delta/2) - \frac{1}{2} f'(x+\Delta/2) \Delta + \frac{1}{8} f''(x+\Delta/2) \Delta^2 - \frac{1}{48} f'''(x+\Delta/2) \Delta^3 + \dots \\
f(x+\Delta) = f(x+\Delta/2) + \frac{1}{2} f'(x+\Delta/2) \Delta + \frac{1}{8} f''(x+\Delta/2) \Delta^2 + \frac{1}{48} f'''(x+\Delta/2) \Delta^3 + \dots
\end{gather*}
Subtracting these and solving for $f'(x+\Delta x/2)$ give, in turn,
\begin{gather*}
f(x+\Delta) - f(x) = f'(x+\Delta/2) \Delta + \frac{1}{24} f'''(x+\Delta/2) \Delta^3 + \dots \\
f'(x+\Delta/2) = \frac{f(x+\Delta) - f(x)}{\Delta} - \frac{1}{24} f'''(x+\Delta/2) \Delta^2 + \dots
\end{gather*}
as desired.

\prob{3}

\mfile{heatwithloops.m}

\prob{4}  If $\Delta t < 0$ then $\mu = D \Delta t / \Delta x^2 < 0$ also.  Then $T_j^{n+1} = \mu T_{j+1}^n + (1 - 2 \mu) T_j^n + \mu T_{j-1}^n$ writes $T_j^{n+1}$ as a linear combination of the three ``old'' values, but with the coefficients of $T_{j+1}^n$ and $T_{j-1}^n$ both negative.  Though the sum of coefficients is still one (i.e.~$\mu + (1-2\mu) + \mu = 1$), this is not an average for any $\Delta t < 0$.

On the other hand, running
\begin{verbatim}
  >> heat(1.0,30,30,-0.001,20);
\end{verbatim}
which goes backward in time from $t=0$ to $t=-0.020$, produces an obviously nonsensical distribution of heat (not shown).

\prob{5}  Rewriting (35) for $T_j^{n+1}$ gives
    $$T_j^{n+1} \stackrel{\ast}{=} T_j^n + \mu T_{j+1}^{n+1} - 2 \mu T_j^{n+1} + \mu T_{j-1}^{n+1}$$
or
    $$T_j^{n+1} = \frac{1}{1 + 2\mu} T_j^n + \frac{\mu}{1 + 2\mu} T_{j+1}^{n+1} + \frac{\mu}{1 + 2\mu} T_{j-1}^{n+1}.$$
In the last form we have written $T_j^{n+1}$ as an average of neighboring points.  That is, the coefficients are positive and add to one.  This allows a ``maximum principle'' proof of unconditional convergence, and thus unconditional stability; see reference [37] (= Morton \& Mayers).

An alternative proof of unconditional stability uses a von Neumann-Fourier argument [37].  This argument examines the evolution of waves $e^{i\omega x}$ on the space-time grid:
    $$T_j^n \stackrel{\dagger}{=} \lambda^n e^{i\omega (j\Delta x)}.$$
We are interested in how the growth/decay rate $\lambda$ depends on the spatial frequency $\omega$.  Numerical stability is the statement ``all waves decay,'' that is, $|\lambda|<1$ for all $\omega$.  Unconditional numerical stability is when this occurs regardless of the values of $\Delta t$ and $\Delta x$, that is, regardless of the value of $\mu$.

In the case of the current implicit scheme, we substitute \emph{ansatz} $\dagger$ into $\ast$ and then cancel as much as possible:
    $$\lambda = 1 + \mu \lambda e^{i\omega\Delta x} - 2 \mu \lambda + \mu \lambda e^{-i\omega\Delta x}$$
or $\lambda = 1 - 2 \mu \lambda + 2 \mu \lambda \cos(\omega\Delta x)$ by recalling $\cos z = (e^{iz} + e^{-iz}) / 2$.  Solving for $\lambda$ gives
    $$\lambda = \frac{1}{1 + 4 \mu \sin^2(\omega\Delta x/2)}$$
because $2 \sin^2 z = 1-\cos(2z)$.  Thus $\lambda > 0$ is obvious and also $\lambda < 1$ is obvious.

This shows that our implicit scheme makes all spatial waves (modes) decay, regardless of the value of $\mu$, and thus regardless of the grid spacings $\Delta t$ and $\Delta x$.  This is unconditional stability.

\prob{6}  (a) The scheme is
    $$\frac{u_j^{n+1} - u_j^n}{\Delta t} = D_0 \frac{u_{j+1}^n - 2 u_j^n + u_{j-1}^n}{\Delta x^2} + E_0 \frac{u_{j+1}^n - u_{j-1}^n}{2\Delta x}.$$
Define constants $\mu = D_0 \Delta t/\Delta x^2$ and $\nu = E_0 \Delta t/\Delta x$.  Note $\mu>0$ but $\nu$ can be of either sign.  Solving for $u_j^{n+1}$ and collecting terms gives
    $$u_j^{n+1} = \left(\mu - \frac{\nu}{2}\right) u_{j-1}^n + \big(1-2\mu\big) u_j^n + \left(\mu + \frac{\nu}{2}\right) u_{j+1}^n.$$

\medskip
\noindent (b)  In order for the last equation to have all nonnegative coefficients, both $\mu - \frac{\nu}{2} \ge 0$ and $\mu + \frac{\nu}{2} \ge 0$ are required, as well as $1-2\mu \ge 0$.  The last condition is the familiar time-step restriction from the notes, namely $\Delta t \le \Delta x^2 / (2 D_0)$.

The other two conditions are new.  They can be combined into one inequality, with several equivalent forms:
    $$\mu - \frac{|\nu|}{2} \ge 0  \quad\iff\quad  \frac{D_0 \Delta t}{\Delta x^2} \ge \frac{|E_0| \Delta t}{2\Delta x}  \quad\iff\quad  \frac{D_0}{\Delta x} \ge \frac{|E_0|}{2}  \quad\iff\quad  \Delta x \le \frac{2 D_0}{|E_0|}$$
The last form gets to the point: if $D_0/|E_0|$ is small then we \emph{must use a fine grid just to have a stable calculation}.  This is not a time-step restriction; if $D_0/|E_0|$ is small enough we might not even have enough computer memory to do a single calculation!

In summary, if advection dominates diffusion ($|E_0| \gg D_0$) then our explicit, centered scheme will be difficult to use.  It will require a certain minimally-fine spatial resolution just to be stable, and then the diffusive time-step restriction may force unpleasantly short time steps.

\medskip
\noindent (c)  We calculate with the product rule, including a reminder about gradient and divergence in two variables,
\begin{align*}
\Div \left(D \grad u\right) &= \Div \left((D\,u_x,D\,u_y)\right) = (D\,u_x)_x + (D\,u_y)_y \\
  &= D\,u_{xx} + D_x\,u_x + D\,u_{yy} + D_y\,u_y = D (u_{xx} + u_{yy}) + (D_x,D_y)\cdot(u_x,u_y) \\
  &= D \grad^2 u + \grad D \cdot \grad u.
\end{align*}
This means we can write the diffusion PDE
    $$u_t = \Div \left(D \grad u\right)$$
 in diffusion-advection form as
    $$u_t = D \grad^2 u + \grad D \cdot \grad u.$$
That is, we could write the diffusion PDE with separate diffusion part ($D \grad^2 u$) and advection part ($\grad D \cdot \grad u$), where the advection speed is $|\grad D|$.

\medskip
\noindent (d)  If we expand ``$\Div \left(D \grad u\right)$'' as in (c) then the strength of the advection relates to the \emph{variability} in $D$, not its magnitude.  This is not a good idea!  That is because $D$ may be small but irregular, with large slopes---in fact it is exactly this way in the SIA context---and then the stability of our scheme is controlled by the steepness of $D$.  Having the time steps controlled by the magnitude of $D$ is o.k.~because it is this magnitude that sets the time scale for the physical process of diffusion.

The explicit centered scheme used in the notes (see \texttt{diffusion.m}) does not require expanding the diffusion PDE into diffusion-advection form.  Instead we evaluate the diffusivity on the staggered grid, and thereby have a centered scheme allowing variability in $D$, but where the only stability restriction depends on the maximum magnitude of $D$ and not its variability.


\prob{7}  Suppose $s = t^{-1/2} x$ and $T(t,x)=t^{-1/2} \phi(s)$.  By the chain rule we can write $T_t$ and $T_{xx}$ with a common power of $t$ and otherwise in terms of $s$:
\begin{align*}
T_t &= - \frac{1}{2} t^{-3/2} \phi(s) + t^{-1/2} \phi'(s) \left(-\frac{1}{2}\right) t^{-3/2} x = - \frac{1}{2} t^{-3/2} \phi(s) - \frac{1}{2} t^{-3/2} s \phi'(s), \\
T_x &= t^{-1/2} \phi'(s) t^{-1/2} = t^{-1} \phi'(s), \\
T_{xx} &= t^{-1} \phi''(s) t^{-1/2} = t^{-3/2} \phi''(s),
\end{align*}
where a prime denotes derivative with respect to $s$.  Thus $T_t = D T_{xx}$ is equivalent to
    $$- \frac{1}{2} t^{-3/2} \phi(s) - \frac{1}{2} t^{-3/2} s \phi'(s) = D t^{-3/2} \phi''(s).$$

Cancelling the common power of $t$ gives a second-order ODE in $s$ for $\phi(s)$:
    $$- \frac{1}{2} \left[\phi(s) + s \phi'(s)\right] = D \phi''(s).$$
But this can be solved because the left side is a derivative, namely $\phi(s) + s \phi'(s) = \left(s \phi(s)\right)'$, so the second-order ODE is $- \frac{1}{2} \left(s \phi(s)\right)' = D \phi''(s)$ or
    $$- \frac{1}{2} s \phi(s) = D \phi'(s) + C_0$$
for some constant $C_0$.  Setting $s=0$ and noting that smoothness requires zero slope at the origin (i.e.~$\phi'(0)=0$) so $C_0=0$.  Now we have a first-order ODE which is separable
    $$- \frac{1}{2D} s \,ds = \frac{d\phi}{\phi}.$$
Its integral gives
    $$- \frac{1}{4D} s^2 + C_1 = \ln|\phi|$$
or, after exponentiating,
    $$\phi(s) = A e^{- s^2/(4D)}$$
for some constant $A = \pm e^{C_1}$.

At this point we have our similarity solution, though with an unknown constant:
    $$T(t,x) = t^{-1/2} \phi(s) = A t^{-1/2} e^{- x^2/(4Dt)}.$$
The constant must be set, whatever the details, by the fact that the initial condition is a \emph{unit} amount of heat: $T(0,x) = \delta(x)$, the Dirac delta function with $\int_{-\infty}^\infty \delta(x)\,dx = 1$.  One way to use this information is to note that the heat equation is the conservation of energy equation, so the total heat energy $\int_{-\infty}^\infty T(t,x)\,dx$ is independent of $t$.  Thus we can set $A$ by noting $T(1,x)$ has the same total energy as the initial amount of heat:
    $$\int_{-\infty}^\infty T(1,x)\,dx = \int_{-\infty}^\infty A e^{- x^2/(4D)}\,dx = 1.$$
Recall, or use a reference to learn that, $\int_{-\infty}^\infty e^{- u^2}\,du = \sqrt{\pi}$.  It follows that
    $$1 = \int_{-\infty}^\infty A e^{- x^2/(4D)}\,dx = A \sqrt{4D} \int_{-\infty}^\infty e^{- u^2}\,du = A \sqrt{4\pi D}$$
by substitution $u=x/\sqrt{4D}$.  Thus $A = (4\pi D)^{-1/2}$ and $T(t,x) = (4\pi D t)^{-1/2} e^{- x^2/(4Dt)}$ as claimed.

\prob{8}

\mfile{verifyheat.m}

\prob{9}  This is an unsolved problem.

\prob{10}  \emph{Though this problem only requires the ability to differentiate, it may be the hardest one in this exercise set!}  For polar coordinates, recall that if $f=f(r)$ then $\grad f = f_r \hat{\mathbf{r}}$, and if $\mathbf{X}=\alpha \hat{\mathbf{r}}$ then $\Div \mathbf{X} = \frac{1}{r} \left(r \alpha\right)_r$.

It helps to clean up the formula before differentiating.  Define new variables $\tau = t/t_0$ and $\rho = r/R_0$ so
    $$H = H_0 \frac{1}{\tau^{1/9}} \left[1 - \left( \frac{\rho}{\tau^{1/18}} \right)^{4/3}\right]^{3/7}.$$
Then equation (26) in the $n=3$ case is equivalent to
    $$\frac{1}{t_0} H_\tau = \frac{\Gamma}{R_0^4} \frac{1}{\rho} \left(\rho H^5 H_\rho^3\right)_\rho.$$
By the formula for $t_0$ just after equation (39), this is equivalent to
    $$18 H_0^7 \left(\frac{4}{7}\right)^3 H_\tau \stackrel{\ast}{=} \frac{1}{\rho} \left(\rho H^5 H_\rho^3\right)_\rho.$$
We will compute the sides of $\ast$ and show they are the same.

First,
    $$H_\tau = H_0 \left(-\frac{1}{9}\right) \tau^{-10/9} \left[1 - \left( \frac{\rho}{\tau^{1/18}} \right)^{4/3}\right]^{3/7} + H_0 \tau^{-32/27} \left(\frac{2}{63}\right) \rho^{4/3}  \left[1 - \left( \frac{\rho}{\tau^{1/18}} \right)^{4/3}\right]^{-4/7}.$$
On the other hand,
    $$H_\rho = - H_0 \left(\frac{4}{7}\right) \tau^{-5/27} \rho^{1/3} \left[1 - \left( \frac{\rho}{\tau^{1/18}} \right)^{4/3}\right]^{-4/7}$$
so
    $$\rho H^5 H_\rho^3 = -\rho^2 H_0^8 \left(\frac{4}{7}\right)^3 \tau^{-30/27} \left[1 - \left( \frac{\rho}{\tau^{1/18}} \right)^{4/3}\right]^{3/7}.$$
Recalling $\ast$, we calculate
\begin{align*}
\left(\rho H^5 H_\rho^3\right)_\rho &= - H_0^8 \left(\frac{4}{7}\right)^3 \tau^{-30/27} \left\{2 \rho \left[1 - \left( \frac{\rho}{\tau^{1/18}} \right)^{4/3}\right]^{3/7} + \rho^2 \left(\frac{3}{7}\right) \left[1 - \left( \frac{\rho}{\tau^{1/18}} \right)^{4/3}\right]^{-4/7}  \left(\frac{-4}{3}\right) \frac{\rho^{1/3}}{\tau^{2/27}}\right\},
\end{align*}
so, with some simplification,
\begin{align*}
\frac{1}{\rho} \left(\rho H^5 H_\rho^3\right)_\rho &= - H_0^8 \left(\frac{4}{7}\right)^3 \tau^{-10/9} \left\{ 2 \left[1 - \left( \frac{\rho}{\tau^{1/18}} \right)^{4/3}\right]^{3/7} - \left(\frac{4}{7}\right) \frac{\rho^{4/3}}{\tau^{2/27}} \left[1 - \left( \frac{\rho}{\tau^{1/18}} \right)^{4/3}\right]^{-4/7} \right\}.
\end{align*}

Comparing $H_\tau$ above to the last expression, this is starting to look good!  We have calculated the right-hand side of $\ast$.  The left side is
    $$18 H_0^7 \left(\frac{4}{7}\right)^3 H_\tau = - H_0^8 \left(\frac{4}{7}\right)^3 \tau^{-10/9} \left\{ 2 \left[1 - \left( \frac{\rho}{\tau^{1/18}} \right)^{4/3}\right]^{3/7} - \left(\frac{4}{7}\right) \frac{\rho^{4/3}}{\tau^{2/27}} \left[1 - \left( \frac{\rho}{\tau^{1/18}} \right)^{4/3}\right]^{-4/7} \right\}.$$
This is the same as our expression for the right-hand side.  Thus the Halfar solution is indeed a solution to the SIA equation.\footnote{Of course, P.~Halfar did not find the solution this way!  One does not just guess such a formula and then check it.  Instead, he or she started with a similarity variable $s = t^\beta r$ and looked for an ODE which the ``self-similar shape'' $\phi(s)$ satisfied.  See exercise \textbf{7} for the heat equation version of such a calculation.  See also reference [10] = Bueler et al.~(2005).}


\prob{11}  The time-dependent margin radius of the Halfar solution is $R(t) = R_0 (t/t_0)^{1/18}$, an expression which comes from setting $H(t,r)=0$ and solving.  The volume at time $t$ is found by integrating out to this radius.  Using polar coordinates and successive substitutions $s = (t/t_0)^{1/18}$ and then $u=r/(R_0 s)$ we have
\begin{align*}
V(t) &= \int_0^{2\pi} \int_0^{R(t)} H(t,r) \,r\,dr\,d\theta \\
     &= 2\pi \int_0^{R_0 (t/t_0)^{1/18}} H_0 \left(\frac{t_0}{t}\right)^{1/9} \left[1 - \left(\left(\frac{t_0}{t}\right)^{1/18} \frac{r}{R_0}\right)^{4/3}\right]^{3/7} \,r\,dr \\
     &= 2\pi H_0 R_0^2 \int_0^1 \left(1 - u^{4/3}\right)^{3/7} \,u\,du.
\end{align*}
At this point we are done!  The last expression is not fully-computed yet,\footnote{The integral can be easily approximated numerically: $\int_0^1 (1 - u^{4/3})^{3/7} \,u\,du \approx 0.31422$.  Thus $V(t) = C H_0 R_0^2$ for a value $C\approx 2$ which can be computed accurately to many digits, if desired.} but it is independent of $t$.
%>> quadl(@(u) u.*(1 - u.^(4/3)).^(3/7),0,1)
%ans =  0.31422
%>> quad(@(u) u.*(1 - u.^(4/3)).^(3/7),0,1)
%ans =  0.31422

\prob{12}  Based on introducing errors big and small in \texttt{siaflat.m}, I believe this claim.  But then, I would.  It is my claim.

\prob{13}  The numerical thickness error plot shows that all of the error is concentrated near the ice sheet margin.  This is because the margin is steep, that is, the gradient of thickness and surface elevation has unbounded magnitude at the margin.  In fact, merely interpolating the exact margin shape onto the grid introduces errors of the same magnitude as the numerical solution error, a claim easily tested using the Halfar solution; see [10] = Bueler et al.~(2005).  Furthermore, \emph{any} ice flow model (e.g.~SIA to Stokes) will produce an ice sheet shape with a steep margin, because the physics leads to such shapes.  Thus this character of the numerical thickness error is universal across models, and unavoidable.\footnote{Assuming the numerical method does not know the asymptotic shape of the margin, which is non-polynomial, and which is not smooth at the free boundary.  Thus any finite difference, finite element, finite volume, spectral, or hp-finite element model will produce thickness error of the same character and magnitude as the boring finite difference method shown here.}

\prob{14}  Let's start with the mass continuity equation (25), write the vertically-averaged velocity as an integral, and then use Leibniz's rule:
\begin{align*}
H_t &= M - \Div \left(\bar{\mathbf{U}} H \right) \\
    &= M - \Div \left(\int_b^h \mathbf{U}(t,x,y,z)\,dz\right) \\
    &= M - \grad h\cdot \mathbf{U}(t,x,y,h) + \grad b\cdot \mathbf{U}(t,x,y,b) - \int_b^h \Div \mathbf{U}(t,x,y,z)\,dz.
\end{align*}
But $M=a-s$ and $H=h-b$.  Also $\Div \mathbf{U} = u_x + v_y = - w_z$ by incompressibility (1).  Thus, changing notation for evaluation at the surface and base and integrating the $z$ derivative,
\begin{align*}
h_t - b_t &= a - s - \grad h\cdot \mathbf{U}\big|_h + \grad b\cdot \mathbf{U}\big|_b + \int_b^h w_z\,dz \\
    &= a - s - \grad h\cdot \mathbf{U}\big|_h + \grad b\cdot \mathbf{U}\big|_b + w\big|_h - w\big|_b.
\end{align*}
We have used the mass continuity equation (25) and incompressibility (1).  Now use the base kinematical equation (41) to simplify the right side:
  $$h_t - b_t = a - \grad h\cdot \mathbf{U}\big|_h + w\big|_h - b_t.$$
Dropping $-b_t$ from both sides we get the surface kinematical equation (40), as desired.

\prob{15}  We start this solution by writing both sides of (44) as derivatives:
    $$\left(2 \,\bar\nu\, H u_x\right)_x = \frac{1}{2} \rho g (1-r) (H^2)_x.$$
Integrating this and recalling $\bar\nu = B |u_x|^{1/n - 1}$ gives
    $$2 B |u_x|^{1/n - 1}\, H u_x = \frac{1}{2} \rho g (1-r) H^2 + C_0$$
for some constant $C_0$.  But (47) applies at the calving front $x=x_c$ so $C_0=0$.  Recalling $B=A^{-1/n}$, cancelling ``$H$'' and simplifying gives
    $$|u_x|^{1/n - 1}\, u_x = \frac{\rho g (1-r) A^{1/n}}{4} H.$$
Note $u_x>0$ for this solution.  We can therefore remove the absolute values, take the $n$th power, and use the defined constant $C_s$ to get
    $$u_x \stackrel{\ast}{=} C_s H^n.$$

On the other hand we can integrate the mass continuity equation $M_0=(uH)_x$ to get $uH = M_0 x + C_1$ for some constant $C_1$.  But at the grounding line $x=x_g=0$ we get $C_1 = u_g H_g$ so
    $$u H \stackrel{\dagger}{=} M_0 x + u_g H_g.$$
That is, the flux increases linearly in $x$ from its specified value at the grounding line.

The ``tricky'' step is to multiply equation $\ast$ by $u^n$ to get
    $$u^n u_x = C_s (uH)^n.$$
Now we substitute $\dagger$ and recognize that we have a separable ODE for $u(x)$:
    $$u^n du = C_s (M_0 x + u_g H_g)^n dx.$$
Integrating from $x_g=0$ to $x$ gives
    $$\frac{u(x)^{n+1}-u_g^{n+1}}{n+1} = \frac{C_s}{M_0 (n+1)} \left((M_0 x + u_g H_g)^{n+1} - (u_g H_g)^{n+1}\right).$$
Solving for $u(x)$ gives the velocity part of the van der Veen solution.  Solving $\dagger$ for $H(x)$ gives the thickness part of the van der Veen solution.

\prob{16}  \emph{I have decided this is not a good exercise.  It will be deleted from future versions of the notes.}

\prob{17}  From (5), by taking norms (second invariants) we have a scalar equation
    $$2 |Du|^2 = Du_{ij} Du_{ij} = (A \tau^2 \tau_{ij})\,(A \tau^2 \tau_{ij}) = A^2 \tau^4 \tau_{ij} \tau_{ij} = 2 A^2 \tau^6$$
so $|Du|=A\tau^3$, as claimed.  Solving for $\tau$ gives $\tau = A^{-1/3} |Du|^{1/3}$.  Again using (5) we get
    $$Du_{ij} = A \left(A^{-1/3} |Du|^{1/3}\right)^2 \tau_{ij} = A^{1/3} |Du|^{2/3} \tau_{ij}.$$
Solving for $\tau_{ij}$ gives the viscosity form
    $$\tau_{ij} = A^{-1/3} |Du|^{-2/3} Du_{ij}.$$
Comparing to ``$\tau_{ij} = 2 \nu\, Du_{ij}$'' gives the formula for $\nu$ stated just after (17).

\prob{18}  As noted, equation (18) follows from dropping $\tau_{13,x}$ from equation (8), giving $(p+\tau_{11})_z = -\rho g$.  Integrating this we get
    $$p+\tau_{11} = -\rho g z + C_0.$$
We take the normal stress to have zero at the surface: $\sigma_{33} = 0$.  Equivalently $p+\tau_{11}=0$ at the surface because $p+\tau_{11} = p-\tau_{33} = -\sigma_{33}$.  Then $0=-\rho g h + C_0$ so finally $p + \tau_{11} = \rho g (h-z)$, which is equation (19).  Taking the $x$ derivative gives an expression for $p_x$, namely: $p_x = -\tau_{11,x} + \rho g h_x$.  Substituting this into (7) gives
    $$-\tau_{11,x} + \rho g h_x = \tau_{11,x} + \tau_{13,z}$$
or
    $$2 \tau_{11,x} + \tau_{13,z} = \rho g h_x.$$
Now we use viscosity form (17) to get
    $$2 \left(2 \nu Du_{11}\right)_x + \left(2 \nu Du_{13}\right)_z = \rho g h_x.$$
Recalling the formula for the strain rate tensor stated after (5), we have
    $$\left(4 \nu u_x\right)_x + \left(\nu (u_z + w_x)\right)_z = \rho g h_x.$$
Which is the hydrostatic stress balance equation (20).

\end{document}

